%%%%% Preamble
\documentclass[10pt,a4paper,parskip=half,DIV=12]{scrartcl}
% KOMA scrartcl is called for several reasons:
% Nicer page margins without the need of the 'geometry' package
% Perfectly fitting abstract environment
% High customizability
\usepackage{url}
\usepackage{color}
\usepackage{fontspec}
\usepackage{tikz}
\usepackage{listings}
\usepackage{setspace}
\usepackage[colorinlistoftodos]{todonotes}
% Code environment
\lstset{language=Ruby, tabsize=2}
\def\codefont{
  \fontspec{Courier New}
  \fontsize{7pt}{8pt}\selectfont}
\definecolor{codebgcolor}{HTML}{EDEDED}
\newenvironment{code}
{\begin{center}
    \begin{tikzpicture}
      \node [fill=codebgcolor,rounded corners=5pt]
      \bgroup
      \bgroup\codefont
      \begin{tabular}{l}}
      {\end{tabular}
      \egroup
      \egroup;
    \end{tikzpicture}
  \end{center}}
% Use it like:
%\begin{code}
%\begin{lstlisting}
%public class HelloWorld {
%  public static void main(String[] args) {
%    /* ... more code ... */
%  }
%}
%\end{lstlisting}
%\end{code}
%
%I can also include the source code from an external file using -
%
%\begin{code}
%\lstinputlisting[...options]{...filepath}
%\end{code}
% Code environment end
% Bug and notes command
\newcommand{\bug}[2][]
 {\todo[color=red, caption={#2}, #1]
 {\begin{spacing}{0.5}#2\end{spacing}}}
\newcommand{\smalltodo}[2][]
 {\todo[caption={#2}, #1]
 {\begin{spacing}{0.5}#2\end{spacing}}}
% Bug and notes command end

%%% LaTeX Template: Article/Thesis/etc. Including abstract
%%% Source: http://www.howtotex.com/
%%% Feel free to distribute this template, but please keep to referal to http://www.howtotex.com/ here.
%%% February 2011
%%% Modifications on February 2013 - Sten Feldman
\usepackage[latin1]{inputenc}						
% Input encoding
\usepackage{amsmath,amsfonts,amssymb}	% Math packages
\usepackage{multicol}
%%%%% Definitions
%%% Define a new command that prints the title only
\makeatletter							% Begin definition
\def\printtitle{%						% Define command: \printauthor
    {\centering \huge \normalfont \textbf{\@title}\par}}
    % Typesetting
\makeatother							% End definition

\title{Ruby-Xbee}

%%% Define a new command that prints the author(s) only
\makeatletter						% Begin definition
\def\printauthor{%					% Define command: \printtitle
    {\large \@author}}				% Typesetting
\makeatother							% End definition

\author{%
	Sten Feldman \\
	\texttt{exile@chamber.ee}\vspace{40pt}
	}

%%% Headers and footers
\usepackage{fancyhdr}								
% Needed to define custom headers/footers
\pagestyle{fancy}								
% Enabling the custom headers/footers
\usepackage{lastpage}	

% Header (empty)
\lhead{}
\chead{}
\rhead{}
% Footer (you may change this to your own needs)
\lfoot{\footnotesize \texttt{https://github.com/exsilium/ruby-xbee} \textbullet ~Dealings with my nightmares - Ruby-Xbee}
\cfoot{}
\rfoot{\footnotesize page \thepage\ of \pageref{LastPage}} % "Page 1 of 2"
\renewcommand{\headrulewidth}{0.0pt}
\renewcommand{\footrulewidth}{0.4pt}

%%% Change the 'KOMA' sectioning fonts back to standard LaTeX font (this is just a matter of taste)
\setkomafont{sectioning}{\rmfamily\bfseries\boldmath}

%%% Change the abstract environment
\usepackage[runin]{abstract}			% runin option for a run-in title
\renewcommand{\abstractname}{Praeambula}
\setlength\absleftindent{16pt}		% left margin
\setlength\absrightindent{16pt}		% right margin
\abslabeldelim{\quad}						% 
\setlength{\abstitleskip}{-10pt}
\renewcommand{\abstracttextfont}{\small \slshape}	% slanted text

%%% Start of the document
\begin{document}
%%% Top of the page: Author, Title and Abstact
\begin{minipage}{0.35\linewidth}
	\begin{flushright}
		\printauthor
	\end{flushright}
\end{minipage} \hspace{0pt}
%
\begin{minipage}{0.02\linewidth}
	\rule{3pt}{205pt}
\end{minipage} \hspace{0pt}
%
\begin{minipage}{0.63\linewidth}
\printtitle 
\vspace{5pt}
\begin{abstract}
You stumbled upon something and are figuring where are you and what is this? WHAT IS THIS!?! As I started to play with Ruby-Xbee and decided to make it work for myself I thought that sooner or later someone might stumble upon it and also dig into the documentation. Well, this is \textbf{not} it. The main Ruby-Xbee gem is documented using RDoc\cite{rdoc} and compiled version is also included in the source code. However, I felt that although RDoc is good for documenting the source itself it's not that good for keeping up with other stuff that happens in the background and which can potentially be equally important in the long run. This article is a compilation of random things and is not meant to be read. But it can potentially be a source of good knowledge - who knows, you might even benefit reading some of this. If you are asking in your mind, why the hell \LaTeX ? Why not? Perhaps I'm up to drawing something...
\end{abstract}
\end{minipage}
% Add some vertical spacing to seperate the abstract from the rest of the article
\vspace{20pt}
% Special character reminder: # $ % & ~ _ ^ \ { }
\setlength{\parindent}{0pt}
\section{Hardware for testing}
\begin{table}[ht]
\caption{Xbees used for testing}
\centering
\begin{tabular}{c c c c}
\hline\hline
Model & Operating Model & Function Set & Firmware \\ [0.5ex] % inserts table %heading
\hline
XB24-Z7WIT-004 & API & Zigbee Coordinator API & 21A7 \\ [1ex]
\hline
\end{tabular}
\label{table:nonlin}
\end{table}

\section{Known Errors and Issues}
\subsection{No more config/xbeeconfig.rb}\smalltodo[noline]{Find replacement solution for xbeeconfig.rb}
The config/xbeeconfig.rb was removed by Mike Ashmore and for good reason. The config had needed configuration detailing where and how to read the XBee module, but it wasn't the best solution especially as it ruined a good way to Gemify the Ruby::XBee code.

\noindent The commit code reads: \textit{"Removed conf/xbeeconfig.rb - it's not a good way to do this. Of cours… …e now I need to figure out what \textbf{is} a good way to do this.} However, seems there was no good solution as currently the values are hardcoded to \textbf{bin/ruby-xbee.rb}.

\subsection{Better exception handling}
\smalltodo[noline]{Implement critical exception handling for proper functioning}
Currently ruby-xbee is very easily choked. Needs to be fixed.

\section{Testing}
\subsection{Wha..?}
Ruby in general is quite famous for it's Test::Unit but also for Rspec and Cucumber - in short, it's a thing to do when developing Ruby/Rails app and organizing an orchestra of good tests can keep headache away later on.

\section{What I'm in the progress of doing}
\begin{itemize}
\item Fixing for Bundle and Rake, integration with Travis and Awesome testing!
\end{itemize}

\section{Memorable quotes picked up along the Journey}
\epigraph{Practically all the software in the world is either broken or very difficult to use. So users dread software. They've been trained that whenever they try to install something, or even fill out a form online, it's not going to work. \textit{I} dread installing stuff and I have a Ph.D. in computer science.}{Paul Graham, \textit{Founders at Work} \cite{rubytutorial}}

\todototoc
\listoftodos
\bibliographystyle{plainurl}
\bibliography{ruby-xbee}
\end{document}