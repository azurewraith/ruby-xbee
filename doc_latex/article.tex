%%% LaTeX Template: Article/Thesis/etc. Including abstract
%%% Source: http://www.howtotex.com/
%%% Feel free to distribute this template, but please keep to referal to http://www.howtotex.com/ here.
%%% February 2011
%%% Modifications on February 2013 - Sten Feldman
\usepackage[latin1]{inputenc}						
% Input encoding
\usepackage{amsmath,amsfonts,amssymb}	% Math packages
\usepackage{multicol}
%%%%% Definitions
%%% Define a new command that prints the title only
\makeatletter							% Begin definition
\def\printtitle{%						% Define command: \printauthor
    {\centering \huge \normalfont \textbf{\@title}\par}}
    % Typesetting
\makeatother							% End definition

\title{Ruby-Xbee}

%%% Define a new command that prints the author(s) only
\makeatletter						% Begin definition
\def\printauthor{%					% Define command: \printtitle
    {\large \@author}}				% Typesetting
\makeatother							% End definition

\author{%
	Sten Feldman \\
	\texttt{exile@chamber.ee}\vspace{40pt}
	}

%%% Headers and footers
\usepackage{fancyhdr}								
% Needed to define custom headers/footers
\pagestyle{fancy}								
% Enabling the custom headers/footers
\usepackage{lastpage}	

% Header (empty)
\lhead{}
\chead{}
\rhead{}
% Footer (you may change this to your own needs)
\lfoot{\footnotesize \texttt{https://github.com/exsilium/ruby-xbee} \textbullet ~Dealings with my nightmares - Ruby-Xbee}
\cfoot{}
\rfoot{\footnotesize page \thepage\ of \pageref{LastPage}} % "Page 1 of 2"
\renewcommand{\headrulewidth}{0.0pt}
\renewcommand{\footrulewidth}{0.4pt}

%%% Change the 'KOMA' sectioning fonts back to standard LaTeX font (this is just a matter of taste)
\setkomafont{sectioning}{\rmfamily\bfseries\boldmath}

%%% Change the abstract environment
\usepackage[runin]{abstract}			% runin option for a run-in title
\renewcommand{\abstractname}{Praeambula}
\setlength\absleftindent{16pt}		% left margin
\setlength\absrightindent{16pt}		% right margin
\abslabeldelim{\quad}						% 
\setlength{\abstitleskip}{-10pt}
\renewcommand{\abstracttextfont}{\small \slshape}	% slanted text

%%% Start of the document
\begin{document}
%%% Top of the page: Author, Title and Abstact
\begin{minipage}{0.35\linewidth}
	\begin{flushright}
		\printauthor
	\end{flushright}
\end{minipage} \hspace{0pt}
%
\begin{minipage}{0.02\linewidth}
	\rule{3pt}{205pt}
\end{minipage} \hspace{0pt}
%
\begin{minipage}{0.63\linewidth}
\printtitle 
\vspace{5pt}
\begin{abstract}
You stumbled upon something and are figuring where are you and what is this? WHAT IS THIS!?! As I started to play with Ruby-Xbee and decided to make it work for myself I thought that sooner or later someone might stumble upon it and also dig into the documentation. Well, this is \textbf{not} it. The main Ruby-Xbee gem is documented using RDoc\cite{rdoc} and compiled version is also included in the source code. However, I felt that although RDoc is good for documenting the source itself it's not that good for keeping up with other stuff that happens in the background and which can potentially be equally important in the long run. This article is a compilation of random things and is not meant to be read. But it can potentially be a source of good knowledge - who knows, you might even benefit reading some of this. If you are asking in your mind, why the hell \LaTeX ? Why not? Perhaps I'm up to drawing something...
\end{abstract}
\end{minipage}
% Add some vertical spacing to seperate the abstract from the rest of the article
\vspace{20pt}